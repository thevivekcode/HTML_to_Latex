\documentclass{article}
\usepackage{hyperref}
\usepackage{comment}
\usepackage[utf8]{inputenc}\usepackage[T1]{fontenc}\usepackage{enumitem}\usepackage{graphicx}\title{Sample document}\author{HET}
\begin{document}
\maketitle
\section{CSS}

\par
Lorem ipsum dolor sit amet, consectetuer adipiscing elit.Nulla ut lectus id velit aliquet semper. Proin vitae erat. Duis metus. Namvel nisl.Duis lobortis mi at lorem. Etiam ornare nibh quis eros. Nam magnasem, adipiscing at,porttitor vitae, interdum vitae, elit. Sed turpis mi,tincidunt eget , euismod ac, molestie quis, wisi.

\par
Lorem ipsum dolor sit amet, consectetuer adipiscing elit.Nulla ut lectus id velit aliquet semper. Proin vitae erat. Duis metus. Namvel nisl. Duis lobortis mi at lorem. Etiam ornare nibh quis eros. Nam magnasem, adipiscing at, porttitor vitae, interdum vitae, elit. Sed turpis mi,tincidunt eget, euismod ac, molestie quis, wisi.

\par
Lorem ipsum dolor sit amet, consectetuer adipiscing elit.Nulla ut lectus id velit aliquet semper. Proin vitae erat. Duis metus. Namvel nisl. Duis lobortis mi at lorem. Etiam ornare nibh quis eros. Nam magnasem, adipiscing at, porttitor vitae, interdum vitae, elit. Sed turpis mi,tincidunt eget, euismod ac, molestie quis, wisi. Praesent nisl pede,hendrerit semper, accumsan ac, consequat id, nibh.

\par
Lorem ipsum dolor sit amet, consectetuer adipiscing.Nulla ut lectus id velit aliquet semper. Proin vitae erat. Duisvel nisl. Duis lobortis mi at lorem. Etiam ornare nibh quis eros.sem, adipiscing at, porttitor vitae, interdum vitae, elit.

\par


Div line 1.

Div line 2.

Div line 3

\par


\begin{comment}
 little comment: <p>&#x03C0; &#960;</p>
\end{comment}

\section{Special symbols}

\section{Greek symbols}

\par
\textt{Alpha}
Hi  \alpha Hello  \beta  \gamma  \pi  \Alpha

\section{LaTeX chars}

\par
\{ \} \_ \^ @ $ \\ \% \~ \#

\section{LaTeX commands in HTML}

\par
It's easy to include LaTeX commands in HTML comments.
\begin{comment}
 latex:
\LaTeX{} greets you.

\end{comment}


\section{Different font styles and sizes.}

\par
Lorem ipsum
{\fontsize{7}{8}\selectfont dolor}
sit amet, \textit{consectetuer}
adipiscing elit.Nulla ut \textbf{lectus}
id velit aliquet semper. \textt{Proin vitae}
erat. Duis metus. Namvel nisl. Duis
{\fontsize{4}{5}\selectfont lobortis}
mi at
{\fontsize{1}{1}\selectfont lorem}
.
\href{None}{}
\section{Images}

\par

\begin{comment}
latex: \LaTeX
\end{comment}
supports only JPG and PNG images.

\par


\begin{center}\includegraphics{marley.jpg}\end{center}

\par


\par
\includegraphics{logo.png}

\section{Tables}
1 11 hgf2hgfhf1 32 12 22 33 13 23 3
\newline
Sparta Praha28Slovan Liberec25Dukla Praha24Slavia Praha20
\section{Subscript, superscript}

\par
H\_{2}O, E = mc\^{2}

\section{Hyperlinks}

\par
I study at \href{http://www.mff.cuni.cz/}{UK MFF}. Andwhat about \href{#img}{images}?

\section{Some texts}

\par


\begin{center}They went in single file, running like hounds on a strong scent,and an eager light was in their eyes. Nearly due west the broadswath of the marching
{\fontsize{4}{5}\selectfont Orcs tramped}
its ugly slot; the sweet grassof Rohan had been bruised and blackened as they passed.\end{center}

\par


\par
John said, I saw Lucy at lunch, she told

\section{Lists and definitions}
\begin{description}[style=unboxed, labelwidth=\linewidth, font =\sffamily\itshape\bfseries, listparindent =0pt, before =\sffamily]\item[Dweeb]

young excitable person who may matureinto a \{em Nerd}
or \{em Geek}

\item[Hacker]

a clever programmer
\item[Nerd]

technically bright but socially inept person
\end{description}

\par
In this section, we discuss the lesser known forest elephants....this section continues...

\section{Habitat}

\par
Forest elephants do not live in trees but among them....this subsection continues...

\subsection{Habitat}

\par
Forest elephants do not live in trees but among them....this subsection continues...  \textbf{AND A LINE FOLLOWS}


\section{List}

\begin{itemize}
\item ... Level one, number one...

\begin{enumerate}
\item ... Level two, number one...

\item ... Level two, number two...

\begin{enumerate}
\item ... Level three, number one...

\end{enumerate}

\item ... Level two, number three...

\end{enumerate}

\item ... Level one, number two...

\end{itemize}
&helloooooooooo
\end{document}
