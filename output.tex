\documentclass{article}
\usepackage{hyperref}
\usepackage{comment}
\usepackage[utf8]{inputenc}
\usepackage[T1]{fontenc}
\usepackage{enumitem}
\usepackage{graphicx}
\title{ COP 701  HTML to LaTeX Converter 2019MCS2574}\author{Vivek Singh}
\begin{document}
\maketitle
\section{ Details of lex and yacc operations : }
\subsection{ LEX file}
\begin{itemize}
\item Have used \textbf{flex} as tool for scanning the HTML file and generating tokens. 
\item The \textbf{ lexer.l } file contains the code for lex operations. 
\item The lex file uses \textbf{caseless option} to tackle the case insenstivity of HTML. Have defined \textbf{spac, special, word, text \&\  greek} type of regex to store information in tags. Some tags have attributes in it hence seperate tokens are generated for each attribute e.g for \textbf{a: name, src, title} etc. All HTML tags are tokenized passes to yacc file.
\end{itemize}
\subsection{ YACC file }
\begin{itemize}
\item Have used \textbf{Bison} for parsing the file and creating equivalent LaTeX document. 
\item The \textbf{ parser.y } file contains the code for yacc operations. 
\item Yacc file returns an AST to \textbf{ ast.cpp } file. Yacc file uses union of char* s and struct node as types of each non terminals. Tokens are used as terminals.
\item Yacc file contains \textbf{Context free grammar (CFG)} for rule definition.
\item ast.h file is included which contains information about structure of AST node. Function \textbf{makenode()} is used to create a new node. Function \textbf{addchildren()} takes two node pointer as argument and makes second node as child of first node. Root is assigned to \textbf{doc\_\ start} which is starting production in grammar. 
\end{itemize}
\section{ AST structure }
\begin{itemize}
\item AST structure is defined in \textbf{ ast.h file.}
\item AST structure contains following in each node
\begin{enumerate}
\item\textbf{nodetype} to store the type of node which will help in recognition of each node.
\item\textbf{string data} to store the data 
\item\textbf{vector children} , stores node which is used as children in Abstract Syntax tree. 
\item\textbf{vector attribute} , is used to store attributes of HTML tags, in pair wise mapping. 
\item\textbf{vector tdata } , to store list of information of a node .
\end{enumerate}
\end{itemize}
\section{ Translating the AST }
\begin{itemize}
\item After successful creation of AST from Grammar rules of yacc file we traverse the AST .
\item Each Html tag data is stored in node with a nodetype for recognition while traversing AST.
\item Tree traversal is DFS type traveral i.e from left to right. 
\item Mapping from each html node is done to equivalent LaTeX tags using a start map and end map for each AST node. 
\end{itemize}
\section{ Programming Language used }
\begin{itemize}
\item\textbf{C++ 11} is used for compiling lex and yacc file 
\item\textbf{Flex} is used for lex operations. 
\item\textbf{Bison} is used for yacc operations. 
\item\textbf{lexer.l} is LEX file
\item\textbf{parser.y} is YACC file.
\item\textbf{ast.h} is header file 
\item\textbf{ast.cpp} contains main function for tree traversal and conversion.
\item\textbf{run.sh} is shell file that takes two argument. First input Second output.
\end{itemize}
\end{document}
